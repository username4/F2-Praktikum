\documentclass[a4paper,ngerman]{scrartcl}

\usepackage{amsmath}
\usepackage{amsfonts}
\usepackage{amssymb}
\usepackage[utf8]{inputenc}
\usepackage{graphicx}
\usepackage[ngerman]{babel}
\usepackage{hyperref}
\usepackage{float}
\usepackage{caption}
\usepackage{subcaption}
\usepackage{multirow}  %for tables
\usepackage{icomma} % Handle german comma as decimal point in numbers
\usepackage{units,siunitx} % Write units with correct spacing
\usepackage{upgreek} % provide non-italic greek letters
\usepackage{url}


% Formatting of table & figure captions
\captionsetup{font={sf,footnotesize},labelfont=bf,textfont=sl,skip=6pt}
\setlength{\abovecaptionskip}{6pt}
\setlength{\belowcaptionskip}{0pt}

\title{Quantenradierer}
\date{\today}
\author{Michel Rausch, Michael Eliachevitch}

\begin{document}

\maketitle
\tableofcontents
\thispagestyle{empty} % remove page number from abstract page
\newpage
\setcounter{page}{1}

\section{Theoretische Überlegungen}
\label{sec:theorie}

\subsection{Interferenz von Wellen, klassisch und quantenmechanisch}
\label{sec:interferenz}

\paragraph{Klassisch Betrachtung:}
\label{par:klass}
In der klassischen Elektrodynamik werden Lichtstrahlen als elektromagnetische Wellen beschrieben, also Wellen im elektromagnetischen Feld,
und sind Lösungen der Maxwellschen Gleichungen. Das elektrische Feld $\vec{E}$ einer ebenen elektromagnetischen Welle breitet sich gemäß der
Gleichung

\begin{equation}
\vec{E}(\vec{x},t) = \vec{E_0} e^{i(\vec{k}\vec{x}-\omega t)}
\end{equation}

aus, wobei die Exponantionfunktion aus Bequemlichkeitsgründen verwendet wird, die Welle aber
als reell angenommen wird. Die Richtung des elektrischen Feldvektors $\vec{E}$ gibt die Polarisation der Welle an.

Die Intensität der Welle ist gegeben über 

\begin{equation}
I(\vec{x},t) = |\vec{E}(\vec{x},t)|^2 = \vec{E}(\vec{x},t) \cdot \vec{E}^*(\vec{x},t)
\end{equation}

Da einzelne Photonen die Energie $E_\gamma = \hbar \omega$ haben, ist die Photonendichte $\eta_{ph}(\vec{x},t)$ gegeben gegeben über

\begin{equation}
  \eta_{ph}(\vec{x},t) = \frac{I(\vec{x},t)}{\hbar \omega} = \frac{|\vec{E}(\vec{x},t)|^2}{\hbar \omega}
\end{equation}

Wenn sich zwei Wellen überlagern, ist die entstehende Welle die Superposition beider Teilwellen.
Wir nehmen nun an, dass sich die Wellen in z-Richtung ausbreiten.
Im Interferometer wird eine ebene Welle mit einem Strahlteiler in zwei Teilwellen geteilt, die dann mit einem Strahlteiler wieder
zusammengeführt werden. Dabei erfährt eine der Wellen eine Laufzeitdifferenz $\Delta z$, was bei der Superposition zu
einem Phasenunterschied $\Delta z k_z$ führt. Aus der Superposition von zwei Wellen mit einem solchen Phasenunterschied
folgt eine Photonendichte

\begin{equation}
  \eta_{ph}(\vec{x},t) = \frac{1}{2} \vec{E}_0 \left( e^{i k_z z} + e^{i k_z (z + \Delta z)} \right) e^{i\omega t},
\end{equation}

was quadratisch zu einer Intensität von 

\begin{equation}
\label{eqn:intensitaet}
  \begin{split}
    I(z,t) &= \frac{|\vec{E_0}|^2 }{4} \left(e^{i k_z z} + e^{i k_z (z + \Delta z)} \right) \left(e^{-i k_z z} + e^{-i k_z (z + \Delta z)} \right)\\
    &= \frac{|\vec{E_0}|^2}{2}(1+\cos(k_z\Delta z) )\\
  \end{split}
\end{equation}

und einer entsprechenden Photonendichte $\eta_{ph}(z,t) = \frac{I(z,t)}{\hbar \omega}$ führt. Der vom Phasenunterschied abhängige
Kosinusterm wird als Interferenzterm bezeichnet, da ohne die Interferenz die Intensität nach der Vereinigung konstant $\frac{|\vec{E_0}|^2}{2}$
wäre. Da am Mach-Zehnder-Interferomter der Phasenunterschied durch die Divergenz an einer Linse entsteht, also vom der Position des Durchgangs
durch die Linse abhängt, sieht man an einem Schirm hinter dem zweiten Strahlteiler konzentrische Ringe, die durch diesen Interferenzterm
entstehen. Diese Ringe sind nur durch die Welleneigenschaften von Licht erklärbar.

\paragraph{Quantenmechanische Betrachtung:}
In der Quantenmechanik betrachtet man Licht zwar als Teilchen, da die Energie und Information gequantelt in Form von Photenen
transportiert, aber schreibt ihnen auch Welleneigenschaften zu. Das liegt daran, dass quantenmechanische Teilchen keine definite
Position haben. Stattdessen werden sie durch eine komplexe Wellenfunktion $\Psi(\vec{x})$ beschrieben, die eine Lösung der
komplexen Schrödingergleichung 
\begin{equation}
  i \hbar \frac{\partial}{\partial t} \Psi(\vec{x},t) = H \Psi(\vec{x},t),
\end{equation}
ist.

Deren Betragsquadrat gibt für jeden Ort im Raum die Aufenthaltswahrscheinlichkeit des Teilchens an:
\begin{equation}
  \rho(\vec{x},t) = |\Psi(\vec{x},t)|^2 = \Psi(\vec{x},t) \cdot \Psi^*(\vec{x},t).
\end{equation}

Bei Photonen ist die Aufenthaltswahrscheinlichkeit proportional zu der Photonendichte $\eta_{ph}$ aus der klassischen Betrachtung von elektromagnetischen Wellen. Die Lösungen der Schrödingergleichung für freie Photonen sind wie in der Elektrodynamik ebene Wellen der Form

\begin{equation}
  \Psi(\vec{x},t) = \Psi_0 e^{i(\vec{k}\vec{x}-\omega t)}.
\end{equation}

Somit lassen sich dir konzentrischen Ringe am Schirm vom Interferometer quantenmechanisch völlig äquivalent zur Elektrodynamik erklären,
mit einem identischen Interferenzterm der durch den Phasenunterschied zustandekommt,
nur dass man statt elektromagnetische Wellen die Wellenfunktion verwendet. \\

Interessant ist, dass auch einzelne Photonen durch eine Wellenfunktion beschrieben werden und es daher auch zu Interferenzeffekton kommt,
wenn man einzelne Teilchen in ein Interferometer schickt.


\subsection{"`Welcher-Weg"'-Information}
\label{sec:welcher-weg}


\clearpage
\section{Das Mach-Zehnder-Interferometer und dessen Aufbau}
\label{sec:mach-zehnder}

\subsection{Aufbau eines einfachen Interferometers}
\label{ssec:interferomter-einfach}

Die theoretischen Überlegungen können in Interferometern geprüft werden. 
In einem einfachem Strahlteilungsinterferometer, wie in Abbildung \ref{fig: Interferometer-einfach} gezeigt. 
Eine planare Welle, hier Licht, wird mit einem halbtransparentem Spiegel in Wellen A und B aufgeteilt. Der Strahl A (rot im Bild) passiert den Strahlteiler und das Interferometer. 
Der zweite wird über Spiegel umgelenkt und erhält so eine längere Laufzeit, mit Gangunterschied $\Delta z$. 
Am zweiten Strahlteiler rekombinieren die Wellen. Dies wurde in \ref{sec:interferenz} beschrieben.

\begin{figure}
\includegraphics[width=0.7\textwidth]{interferomter-einfach.png}
\caption{Einfaches Interferometer, zur Strahlteilung und -rekombination, mit dem Laufzeitunterschied $\Delta z$ [\ref{ref:mappe}]}
\label{fig: Interferometer-einfach}
\end{figure}

\subsection{Quantenradierer}
\label{ssec:quantenradierer}

Man kann man die Eigenschaften von Photonen wie die Polarisation, und somit die Wellenfunktion, ändern. 
Bei anderen Quanten kann auch zum Beispiel der Spin manipuliert werden.

In diesen Versuch wird das Mach-Zehnder-Interferometer, wie in Abbildung \ref{fig:mach-zehnder}, verwendet. Dieses teilt eine eintreffende Welle auf. Die eintreffende Welle ist beschrieben durch ihr elektrisches Feld
 
\begin{equation}
\vec{E_0} = E_0 \ \begin{pmatrix} 1\\ 0 \end{pmatrix}
\end{equation}

Die Teilstrahlen A 	und B werden unterschiedlich polarisiert, mit Polarisationswinkeln von $45^{\circ}$ und $-45^{\circ}$. 
Hieraus ergeben sich

\begin{equation}
\vec{E_A} = \frac{E_0}{2 \sqrt{2}} \ \begin{pmatrix} 1\\ 1 \end{pmatrix} \ e^{i (k_z z -\omega t)}
\end{equation}
und
\begin{equation}
\vec{E_B} = \frac{E_0}{2 \sqrt{2}} \ \begin{pmatrix} 1\\ -1 \end{pmatrix} \ e^{i (k_z (z + \delta z) -\omega t)} .
\end{equation}

Hierbei wurde ohne Einschränkung der Allgemeinheit angenommen, dass sich die Wellen in z-Richtung ausbreiten. 
Der Gangunterschied $\Delta z$ stammt im Experiment von einer Linse, die eine kleine Divergenz der Strahlen verursacht.

\begin{figure}
\includegraphics[width=0.7\textwidth]{mach-zehnder.png}
\caption{Skizze eines Mach-Zehnder-Interferomters mit Polarisatoren ($45^{\circ}$ und $-45^{\circ}$) [\ref{ref:mappe}]}
\label{fig:mach-zehnder}
\end{figure}

Durch die Wahl der Polarisation sind die Wellen A und B offensichtlich orthogonal. Der Interferenzterm aus Gleichung \ref{eqn:intensitaet} verschwindet. Es gilt somit der klassische Fall für die Intensität
\begin{equation}
I(z,t) = \frac{|\vec{E_0}|^2}{2}.
\end{equation}
Es ist somit kein Interferenzmuster sichtbar. 
Ohne Polarisationsfilter wäre ein ringförmiges Interferenzmuster zu erwarten.
Mit einem dritten Polarisator, der vor einem Abbildungsschirm montiert wird, kann man das Muster wieder sichtbar machen. 
Ist der Polarisator auf $0^{\circ}$ gestellt, ergeben sich die Teilwellen als

\begin{equation}
\vec{E_A} = \frac{E_0}{4} \ \begin{pmatrix} 1\\ 0 \end{pmatrix} \ e^{i (k_z z -\omega t)}
\end{equation}
und
\begin{equation}
\vec{E_A} = \frac{E_0}{4} \ \begin{pmatrix} 1\\ 0 \end{pmatrix} \ e^{i (k_z z -\omega t)}.
\end{equation}

Hier verschwindet der Interferenzterm offensichtlich nicht und das Muster wird wieder erkennbar. Gleichung \ref{eqn:intensitaet} bleibt somit

\begin{equation}
I(z,t) = \frac{|\vec{E_0}|^2}{2} \ (1+\cos(k_z\Delta z) ).
\end{equation}

Man sieht also deutlich die Signifikanz der "Welcher-Weg"-Information aus \ref{sec:welcher-weg}.


\section{Versuchsdurchführung}
\label{sec:versuchsdurchfuhrung}








\clearpage
\section{Quellen}
\begin{enumerate}
\item Vorbereitungsmappe \label{ref:mappe}
\end{enumerate}



\end{document}
