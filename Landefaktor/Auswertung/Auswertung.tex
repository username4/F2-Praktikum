\documentclass[a4paper,ngerman]{scrartcl}

\usepackage{amsmath}
\usepackage{amsfonts}
\usepackage{amssymb}
\usepackage[utf8]{inputenc}
\usepackage{graphicx}
\usepackage[ngerman]{babel}
\usepackage{hyperref}
\usepackage{float}
\usepackage{caption}
\usepackage{subcaption}
\usepackage{multirow}  %for tables
\usepackage{icomma} % Handle german comma as decimal point in numbers
\usepackage{units,siunitx} % Write units with correct spacing
\usepackage{upgreek} % provide non-italic greek letters
\usepackage{url}
\usepackage{booktabs}
%\usepackage{subfig}

% Formatting of table & figure captions
\captionsetup{font={sf,footnotesize},labelfont=bf,textfont=sl,skip=6pt}
\setlength{\abovecaptionskip}{6pt}
\setlength{\belowcaptionskip}{0pt}

\title{Landéfaktor\\Versuchsauswertung}
\date{\today}
\author{Michel Rausch, Michael Eliachevitch}

\begin{document}

\maketitle
\tableofcontents
\newpage

\section{Zeitkalibrierung}

\section{Bestimmung der Lebensdauer des Myons}

\section{Bestimmung des Landéfaktors des Myons}
%Am Besten noch was zu g schreiben, in der Vorbereitung war nicht ausreichend erklärt, was es ist.

\subsection{Bedeutung des Landéfaktors}

\section{Diskussion der Ergebnisse}

\section{Quellen}
\begin{enumerate}
\item Vorbereitungsmappe 
\end{enumerate}



\end{document}
