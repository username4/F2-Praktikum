\documentclass[a4paper,ngerman]{scrartcl}

\usepackage{amsmath}
\usepackage{amsfonts}
\usepackage{amssymb}
\usepackage[utf8]{inputenc}
\usepackage{graphicx}
\usepackage[ngerman]{babel}
\usepackage{hyperref}
\usepackage{float}
\usepackage{caption}
\usepackage{subcaption}
\usepackage{multirow}  %for tables
\usepackage{icomma} % Handle german comma as decimal point in numbers
\usepackage{units,siunitx} % Write units with correct spacing
\usepackage{upgreek} % provide non-italic greek letters
\usepackage{url}
%\usepackage{subfig}

% Formatting of table & figure captions
\captionsetup{font={sf,footnotesize},labelfont=bf,skip=6pt}
\captionsetup[sub]{font={sf,footnotesize}} % setting for subcaptions
\sisetup{ locale = DE, % use "," as decimal point instead of "."
per-mode=fraction, % use fractions instead of ^{-1} when doing \si{... \per ...} 
exponent-product={\cdot},% used \cdot in front of 10^x
separate-uncertainty % give out uncertainty with \pm instead of in brackets
} 
\setlength{\abovecaptionskip}{6pt}
\setlength{\belowcaptionskip}{0pt}

\title{Landé-Faktor des Myons\\Versuchsvorbereitung}
\date{\today}
\author{Michel Rausch, Michael Eliachevitch}

\begin{document}

\maketitle
\tableofcontents
\newpage

\section{Theoretische Grundlagen}
\label{sec:theorie}

\subsection{Enstehung von Myonen in Luftschauern}
\label{sec:luftschauer}
% kurz schreiben, dass hochenergetische kosmische teilchen luftschauer erzeugen,
% bei denen durch hadronische wechselwirkungen geladene pionen entstehen, bei deren zerfall myonen entstehen
% vielleicht auch weglassen und erst später beim kapitel zum spin
% lebensdauer von myonen


\subsection{Abbremsung von Myonen in Materie}
\label{sec:wwmitmaterie}
% bei sehr hohen energien (> 1 TeV) bremmstrahlung
% bei niedrigen energien im GeV Bereich Ionisation (Myonen sind 'mips')
% wechselwirkungen mit atomen bei niedrigen energien: 
% niederenergetische myonen wie elektronen in kern gebunden -> schneller zerfall
% antimyonen bilden kernähnliche struktur durch elektroneneinfang -> myonium

\subsection{Polarisation von Myonen}
\label{sec:polarisation}
% schwacher wechselwirkung -> vektorielle kraft und paritätsverletzung
% d.h. masselose teilchen immer linkshändig, antiteilchen rechtshändig
% da pion spin 0 hat und neutrino linkshändig ist, muss positron/antimyon auch linkshändig sein
% wegen schwacher ww wären positronen und antimyonen "lieber" rechtshändig
% da positronen kleinere masse haben als anti-muonen (faktor ~200), ist bei ihnen die rechtshändige komponente stärker unterdrückt
% analog für antineutrinos und elektronen/myuonen
% daher zerfall von piplus/piminus hauptsächlich in muonen und nicht in elektronen/positronen

% zerfall in vorwärts und rückwärtsrichtung
% myon erhält dadurch in vorwärts und rückwärtsrichtung unterschiedliche energien im laborsystem bei gleichen pionenenergien
% unterscheidbar durch polarisation



\subsection{Nachweis des Myonenzerfalls}
\label{sec:nachweis}

% die theorie hiervon kann ich auch noch nicht gut (michael)
% aber müsste nicht viel sein

\subsection{Präzission von Myonen im Magnetfeld}
\label{sec:prazission}
% gyromagnetische verhältnis gamma, landé-faktor g erklären
% präzission im magnetfeld
% am wichtigsten ist eigentlich nur folgende gleichung:
\begin{equation}
g = \frac{\gamma \hbar}{\mu_\mathrm{B}} = \frac{\hbar \omega}{\mu_\mathrm{B}\mathrm{B}}
\end{equation}

\subsection{Messprinzip}
\label{sec:messprinzip}
% zerfall von myonen erfolgt exponentialgesetz
% wie misst man präzission?
% -> angelegtes magnetfeld
% formeln...
\clearpage

\section{Versuchsaufbau und Durchführung}
% von oben nach unten: Szinti 1, Szinti 2, Kupferabsorber in Magnetfeld, Szinti 3
% Koinzindenz und Veto-Schaltung erklären: 
% D.h. Wir nehmen nur die Ereignisse, wo 1 und 2 gleichzeitig ausschlagen (Koinzindenz) und 3 NICHT ausschlägt (Veto bzw Antikoinzindenz)
% erklären was ein diskriminator in der analog-digitalwandlung ist und wie die schwelle (trigger) eingestellt werden muss:
% trigger muss so sein, dass untergrund unterdrückt wird und man nur hauptsignal sieht
% zu hoher trigger verringer unnötig statistik
% trigger für veto (szinti 3) lieber etwas niedriger als zu hoch, damit man keine falschen ereignisse aufnimmt
% was ist ein TAC (zeit-amplituden-converter) und zeiteichung ?

\clearpage
\section{Quellen}
\begin{enumerate}
\item \emph{Einführung in das Kernphysikalische Praktikum} von F. K. Schmidt, 
  Überarbeitung von J. Wolf, Ausgabe September 2009 \label{ref:bb}
\end{enumerate}



\end{document}
