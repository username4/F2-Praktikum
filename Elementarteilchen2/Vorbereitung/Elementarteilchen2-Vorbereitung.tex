\documentclass[a4paper,ngerman]{scrartcl}

\usepackage{amsmath}
\usepackage{amsfonts}
\usepackage{amssymb}
\usepackage[utf8]{inputenc}
\usepackage{graphicx}
\usepackage[ngerman]{babel}
\usepackage{hyperref}
\usepackage{float}
\usepackage{caption}
\usepackage{subcaption}
\usepackage{multirow}  %for tables
\usepackage{icomma} % Handle german comma as decimal point in numbers
\usepackage{units,siunitx} % Write units with correct spacing
\usepackage{upgreek} % provide non-italic greek letters
\usepackage{url}
\usepackage{hepnames} % hier gibt es symbole fuer unterschiedliche elementarteilchen, aber dieses paket gibts nicht im poolraum

%\usepackage{subfig}

% Formatting of table & figure captions
\captionsetup{font={sf,footnotesize},labelfont=bf,textfont=sl,skip=6pt}

\sisetup{locale = DE, % use "," as decimal point instead of "."
  exponent-product={\cdot},% used \cdot in front of 10^x
  separate-uncertainty} % give out uncertainty with \pm instead of in brackets

\setlength{\abovecaptionskip}{6pt}
\setlength{\belowcaptionskip}{0pt}

\title{Elementarteilchen 2\\Vorbereitung}
\date{\today}
\author{Michel Rausch, Michael Eliachevitch}

\begin{document}

\maketitle
\tableofcontents
\newpage

\section{Einleitung}

\section{Das DELPHI Experiment am LEP}
\label{sec:delphi}


\section{Mögliche Zerfälle und Bestimmung der Zerfallsbreiten und Kopplungskonstanten}
\label{sec:zerfaelle}


\section{Das Scannen von Ereignissen mit "`Fireworks"' und Beispielzerfälle}
\label{sec:scannen}

\section{Quellen}
\begin{enumerate}
\item Vorbereitungsmappe 
% Quelle fuer PDG-Angaben: (noch nicht genutzt, daher auskommentiert)
% \item K.A. Olive et al. (Particle Data Group), Chin. Phys. C, 38, 090001 (2014). \label{ref:pdg14}
\end{enumerate}



\end{document}
