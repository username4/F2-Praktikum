\documentclass[a4paper,ngerman]{scrartcl}

\usepackage{amsmath}
\usepackage{amsfonts}
\usepackage{amssymb}
\usepackage[utf8]{inputenc}
\usepackage{graphicx}
\usepackage[ngerman]{babel}
\usepackage{hyperref}
\usepackage{float}
\usepackage{caption}
\usepackage{subcaption}
\usepackage{multirow}  %for tables
\usepackage{icomma} % Handle german comma as decimal point in numbers
\usepackage{units,siunitx} % Write units with correct spacing
\usepackage{upgreek} % provide non-italic greek letters
\usepackage{url}
\usepackage{hepnames} % hier gibt es symbole fuer unterschiedliche elementarteilchen, aber dieses paket gibts nicht im poolraum

%\usepackage{subfig}

% Formatting of table & figure captions
\captionsetup{font={sf,footnotesize},labelfont=bf,textfont=sl,skip=6pt}

\sisetup{locale = DE, % use "," as decimal point instead of "."
  exponent-product={\cdot},% used \cdot in front of 10^x
  separate-uncertainty} % give out uncertainty with \pm instead of in brackets

\setlength{\abovecaptionskip}{6pt}
\setlength{\belowcaptionskip}{0pt}

\title{Elementarteilchen 2\\Vorbereitung}
\date{\today}
\author{Michel Rausch, Michael Eliachevitch}

\begin{document}

\maketitle
\tableofcontents
\newpage

\section{Einleitung}

Nach dem Standartmodell (\textbf{SM}) der Teilchenphysik gibt es zwei große Gruppen der Elementarteilchen.
Eine werden Fermionen genannt und besitzen einen Spin von $\frac{1}{2}$. 
Zu ihnen gehören beispielsweise Elektronen.
Die andere Gruppe sind die Bosonen mit einem Spin von $1$.

Fermionen sind weiterhin gegliedert in Leptonen und Quarks.
Des weiteren sind sie in 3 Generationen unterteilt aus jeweils zwei Quarks und zwei Leptonen. 
Eine Übersicht 
Up- und Down-Quarks bilden mit Elektronen die Grundlage für gewöhnliche Materie.
Diese Teilchen lassen sich in einem Teilchenbeschleuniger erzeugen und untersuchen.

In diesem Versuch werden die Eigenschaften einiger Teilchen untersucht anhand der Daten des DELPHI. 
Auch die Wechselwirkung zwischen Elementarteilchen wird untersucht.
Die Funktionsweise eines Detektors, sowie die Auswertung der entstandenen Daten soll mit diesem Experiment verständlich gemacht werden.



\section{Das DELPHI Experiment am LEP}
\label{sec:delphi}

Der LEP (Large Electron Positron) Collider war ein Teilchenbeschleuniger des CERN.
Er befand sich nahe Genf \SI{100}{\metre} unter der Erde und war \SI{27}{\kilo \metre} lang. 
Mittlerweile ist er vom Large Hadron Collider (LHC) ersetzt wurden.

An vier Stufen wurden Elektronen und Positronen in entgegengesetzte Richtung auf Energien bis zu \SI{100}{GeV} beschleunigt.
Die Strahlen treffen an vier Orten zusammen.
Dort befinden sich Detektoren, um die Kollisionen zu analysieren, darunter der DELPHI (Detector with Lepton, Photon and Hadron Identification), der schematisch in Abbildung \ref{fig:delphi_big} gezeigt ist.

\begin{figure}[tb!]
\centering
\includegraphics[width=\textwidth]{abbildungen/delphi_big.png}
\caption{\textbf{Aufbau des DELPHI-Detektors mit einer sichtbaren Endkappe [\ref{ref:hands-on}].} 
Die Bestandteile des Detektors sind farblich gekennzeichnet und benannt.
Links ist eine der beiden Endkappen zu sehen, die andere wurde zur Übersichtlichkeit ausgeblendet.
Rechts im Bild ist der zylindrische zentrale Teil mit Strahlengang gezeigt. 
}
\label{fig:delphi_big}
\end{figure}

In dem Speicherring werden Elekt






\begin{figure}[tb!]
\centering
\includegraphics[width=0.7\textwidth]{abbildungen/delphi_schichten.png}
\caption{\textbf{Übersicht der Signaturen nachweisbarer Teilchen in den Subdetektoren [\ref{ref:BB}].} 
Photonen durchqueren die Spurkammer und werden erst im ECAL detektiert.
Elektronen und Positronen bilden Spuren in der Spurkammer und verlieren ihre Energie im ECAL.
Myonen bilden Spuren bis in die Myonkammer.
Hadronen lösen verlieren ihre Energie im HCAL.
Neutronen bilden keine Spur in der Spurkammer und sind nicht im ECAL sichtbar, da sie keine Ladung besitzen.
Protonen und geladene Pionen hingegen sind in der Spurkammer, sowie im ECAL sichtbar.
}
\label{fig:delphi_schichten}
\end{figure}


\section{Mögliche Zerfälle und Bestimmung der Zerfallsbreiten und Kopplungskonstanten}
\label{sec:zerfaelle}



\section{Das Scannen von Ereignissen mit "`Fireworks"' und Beispielzerfälle}
\label{sec:scannen}

\section{Quellen}
\begin{enumerate}
\item Blaues Buch \label{ref:BB}
% Quelle fuer PDG-Angaben: (noch nicht genutzt, daher auskommentiert)
% \item K.A. Olive et al. (Particle Data Group), Chin. Phys. C, 38, 090001 (2014). \label{ref:pdg14}
\item \url{http://hands-on-cern.physto.se/hoc_v21en/index.html} (25.1.2015)\label{ref:hands-on}
\end{enumerate}



\end{document}
