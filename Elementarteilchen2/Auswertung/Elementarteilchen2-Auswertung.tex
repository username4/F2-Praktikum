\documentclass[a4paper,ngerman]{scrartcl}

\usepackage{amsmath}
\usepackage{amsfonts}
\usepackage{amssymb}
\usepackage[utf8]{inputenc}
\usepackage{graphicx}
\usepackage[ngerman]{babel}
\usepackage{hyperref}
\usepackage{float}
\usepackage{caption}
\usepackage{subcaption}
\usepackage{multirow}  %for tables
\usepackage{icomma} % Handle german comma as decimal point in numbers
\usepackage{units,siunitx} % Write units with correct spacing
\usepackage{upgreek} % provide non-italic greek letters
\usepackage{url}
\usepackage{hepnames} % hier gibt es symbole fuer unterschiedliche elementarteilchen, aber dieses paket gibts nicht im poolraum
\usepackage{booktabs}
%\usepackage{subfig}

% Formatting of table & figure captions
\captionsetup{font={sf,footnotesize},labelfont=bf,textfont=sl,skip=6pt}

\sisetup{locale = DE, % use "," as decimal point instead of "."
  exponent-product={\cdot},% used \cdot in front of 10^x
  separate-uncertainty} % give out uncertainty with \pm instead of in brackets

\setlength{\abovecaptionskip}{6pt}
\setlength{\belowcaptionskip}{0pt}

\title{Elementarteilchen 2\\Auswertung}
\date{\today}
\author{Michel Rausch, Michael Eliachevitch}

\begin{document}

\maketitle
\tableofcontents
\newpage

% Aufgaben
% - Bestimmung der Verzweigungsverhältnisse beim Z°-Zerfall und der Anzahl der
% verschiedenen Farbladungen der starken Wechselwirkung.
% - Bestimmung der Kopplungskonstanten der starken Wechselwirkung α s mit Hilfe der 3-
% Jet-Rate bei Erzeugung und Zerfall des Z-Bosons.
% - Bestimmung der Anzahl der Neutrino-Generationen im Standardmodell mittels der
% Zerfallsbreite des Z°.

\section{Verzweigungsverhältnisse beim \PZzero-Zerfall und Anzahl der Farbladungen}
\label{sec:verzweigungen}

Es wurden Ereignisse am LEP aus der Onlinequelle \ref{ref:hands-on} mithilfe des webbasiertem Programms analysiert.
Die Daten Lagen in Messreihen von 100 Ereignissen vor.
Aufgeteilt wurde nach \Pelectron\APelectron ,	\Ptauon\APtauon	,	\Pmuon\APmuon , hadronischen 2-Jet-Ereignissen, sowie hadronischen 3-und Mehr-Jet-Ereignissen.
Es wurden 5 Messreihen aufgenommen. 
Bis auf dem Ereignis 197 konnten alle Ereignisse identifiziert werden.
Insgesamt wurden also 499 Ereignisse gezählt.
Für jede Messreihe ist die Anzahl der unterschiedlichen Ereignisse in Tabelle~\ref{tab:count} aufgelistet.
Die 

\begin{table}
\centering
\caption{\textbf{Anzahl der Ereignisse.} 
Gezeigt sind die Häufigkeiten der jeweiligen Zerfallsarten für jede Messreihe mit je 100 Ereignissen. 
Insgesamt wurden 499 Ereignisse identifiziert.
Ein Ereignis (197) konnte nicht zugeordnet werden.
Zusätzlich ist der Anteil und die Gesamtzahl der Zerfallsarten aufgelistet.
}
\begin{tabular}{cccccc}
\toprule
Messreihe			&	\Pelectron\APelectron	&	\Ptauon\APtauon	&	\Pmuon\APmuon	&	3-Jet	&	2-Jet	\\
\midrule
1-100				& 	8	&	3	&	5	&	33	&	51	\\
101-200 (ohne 197)	& 	2	&	2	&	2	&	35	&	58	\\
201-300				&	3	&	1	&	2	&	36	&	58	\\
301-400				&	7	&	1	&	3	&	34	&	55	\\
401-500				&	5	&	2	&	3	&	37	&	53	\\
Gesamt				&	25	&	9	&	15	&	175	&	275	\\
Anteil				&5,0\%	&1,8\%  &3,0\%	&35,1\%	&55,1\%	\\
%(mapcar #'(lambda (x) (/ x 499.0)) '(25 9 15 175 275))
% (0.050100200400801605 0.018036072144288578 0.03006012024048096 0.35070140280561124 0.5511022044088176)
\bottomrule
\end{tabular}
\label{tab:count}
\end{table}


Das Verhältnis von hadronischen zu leptonischen Zerfällen $R$ wurde aus Gleichung~(6) bestimmt.
Mit der ersten Messreihe wurde abgeschätzt, wie viele Ereignisse untersucht werden mussten.
In der ersten Reihe wurden 16 leptonische und 84 hadronische Ereignisse gezählt.
Wird die Zahl der leptonischen Ereignisse $n$ extrapoliert mit einem Faktor $m$, ist der zu erwartende relative Fehler auf die Anzahl der gezählten leptonischen Ereignisse $n \cdot m$
\begin{equation}
\sqrt{\frac{1}{n \cdot m}} ~.
\end{equation}
$m$ gibt die Anzahl der Messreihen an. 
Die Zahl der hadronischen Ereignisse ist $100 - n $, daher ist $R$ nur von $n$ abhängig. 
Damit der relative Fehler unter 15~\% ist, muss $m$ größer als $2,78$ sein.
Es mussten also mindestens 3 Messreihen aufgenommen werden.

Es wurde Leptonuniversalität angenommen, d.h. es galt für die Zerfallsbreiten
\begin{equation}
\label{eq:leptonuniversal}
\Gamma_{l\bar{l}} = \frac{\Gamma_{e\bar{e}} + \Gamma_{\mu\bar{\mu}} + \Gamma_{\tau\bar{\tau}}}{3} ~.
\end{equation}

Die Anzahl der Elektronen-, Myonen- und Tauonenzerfälle betrugen 25, 15 und 9.
Im Mittel sind es 
\begin{equation}
N_{l\bar{l}} = \SI{16+-7}{}
\end{equation}
Die Werte weichen stark voneinander ab, mit einer Standartabweichung von 40~\%.
Durch die Annahme von Leptonuniversalität wurde dieser Fehler nicht fortgepflanzt.

Das Verzweigungsverhältnis ist mit Gl.~\eqref{eq:leptonuniversal}
\begin{equation}
R = \frac{3 \cdot \Gamma_{\mathrm{had}}}{\Gamma_{e\bar{e}} + \Gamma_{\mu\bar{\mu}} + \Gamma_{\tau\bar{\tau}}} ~.
\end{equation}
Da die Zerfallsbreiten proportional zu den Ereigniszahlen sind, konnten letztere verwenden werden, ohne die Zerfallsbreiten zu bestimmen.
Das Verhältnis war damit
\begin{equation}
R = \SI{28+-4}{} ~.
\end{equation}
Der Fehler ergab sich aus dem Fehler für die Anzahl leptonischen Endzustände. 
Dieser lag bei 14~\%, also unter den angestrebten 15~\%.

Aus Gleichung~(8) der Vorbereitung ließ sich die Farbzahl $N_{\mathrm{C}}$ bestimmen mit
\begin{equation}
N_{\mathrm{C}} = \frac{\Gamma_{\mathrm{had}}^{\mathrm{SM}}}{N_u \Gamma_{u \bar{u}}^{\mathrm{SM}} + N_d \Gamma_{d \bar{d}}^{\mathrm{SM}}} ~.
\end{equation}

$N_u = 2$ ist die Anzahl der beobachtbaren "'Up"'-artigen Quarks, $N_d$ die der "'Down"'-artigen.
Die Zerfallsbreite für Hadronen wurde berechnet mit

\begin{equation}
\Gamma_{\mathrm{had}} = R \cdot \Gamma_{l \bar{l}}^{\mathrm{SM}} = \SI{2.3+-0.3}{\GeV} ~.
\end{equation}

Aus dem Blauen Buch [\ref{ref:BB}] wurden die partiellen Zerfallsbreiten $\Gamma_{l \bar{l}}^{\mathrm{SM}} = \SI{83.83}{\MeV}$, $\Gamma_{u \bar{u}}^{\mathrm{SM}} = \SI{98.88}{\MeV}$ und $\Gamma_{d \bar{d}}^{\mathrm{SM}} = \SI{137.48}{\MeV}$ entnommen.
Der Fehler wurde linear aus dem Verzweigungsverhältnis fortgepflanzt.
Die Farbzahl wurde errechnet mit
\begin{equation}
N_{\mathrm{C}} = \SI{4.0+-0.6}{} ~.
\end{equation}

Dies widerspricht mit einer Abweichung von 1.7 Standardabweichungen der erwarteten Farbzahl von 3,
wie sie in der QCD angenommen wird.
Weil die relative Abweichung aufgrund der geringen Ereigniszahl hoch war, könnte es sich um einen statistischen Fehler gehandelt haben. 
Dazu kommt noch ein wahrscheinlich großer systematischer Fehler vom Scannen der Ereignisse 
und Fehler im Entscheiden, zu welchem Zerfall ein Ereignis gehört.
Ein Hiweis darauf ist zum Beispiel, dass unsere Messungen die Leptonenuniversalität nicht bestätigten,
da Elektron-, Myonen- und Tauon-Zerfälle unterschiedlich schwer zu erkennen waren.
Unser Faktor R scheint deutlich zu groß zu sein.
In der Literatur~\ref{ref:pdg14} findet sich ein Wert $\Gamma_{\rm had, Lit.} \SI{1744,4 +- 2}{\mega\electronvolt}$, 
womit folgt

\begin{equation}
  R_{\rm Lit.} = \frac{\SI{1744,4}{MeV}}{\SI{83,83}{MeV}} \approx 21~,
\end{equation}

womit unser Wert für $R$ 33\% zu hoch ist, was sehr gut die zu hohe Farbzahl erklärt.
Womöglich haben einige hochenergetische Tauon-Zerfälle fälschlicherweise für Jet-Ereignisse gehalten.




\clearpage

\section{Bestimmung der Kopplungskonstanten der starken Wechselwirkung}
\label{sec:kopplungskonstante}

Die Kopplungskonstante $\alpha_s$ der starken Wechselwirkung kann
mithilfe von Gleichung~(9) aus der Versuchsvorbereitung bestimmt
werden. Wir haben $N_3(y > y_{\rm cut}) = 175$ 3-Jets gemessen, von
$N_{\rm had} = 450$ Jet-Ereignissen insgesamt, und verwenden eine für
DELPHI bestimmte Konstante $C(y > y_{\rm cut} = 2{,}72$~\ref{ref:BB},
womit die Gleichung~(9) folgendes ergibt:

% (/ 175 (* 450 2.72)) 0.14297385620915032
\begin{equation}
  \frac{N_3(y > y_{\rm cut})}{C(y > y_{\rm cut})\cdot N_{\rm had}} = \frac{175}{2{,}72 \cdot 450} = 0,14 \approx \alpha_s~.
\end{equation}

Der Fehler hierfür folgt über aus dem ausschließlich relativen Fehler von $N_3$, da $N_2$ über $N_2 = N_h - N_3$ 
vollständig mit $N_3$ korreliert ist, wenn man $N_h$ als fest annimmt. 
Damit erhalten wir einen Fehler 

%(/ 0.143 (+ (sqrt 175) 0))
\begin{equation}
  \Delta \alpha_s = \frac{\alpha_s}{\sqrt{N_3}} = 0,01~.
\end{equation}

Somit erhalten wir die Kopplungskonstante der starken Kernkraft

\begin{equation}
  \alpha_s = \SI{0,14 +- 0,01}{}~.
\end{equation}

Die starke Kopplungskonstante ist energieabhängig.
Sie sinkt mit dem Logarithmus der Schwerpunktsenergie $\sqrt{s}$ und lässt sich über
folgende empirische Formel aus der Literatur~\ref{ref:hyperphysics} berechnen:
\begin{equation}
  \alpha_s = \frac{12 \pi}{33 - 2 \cdot (N_u + N_f) \cdot \log\left(\frac{s}{\Lambda^2}\right)}~,
\end{equation}
wobei der empirische Parameter $\Lambda = \SI{0.2}{\giga\electronvolt}$ ist. 
Da die Schwerpunktsenergie unserer Ereignisse $\sqrt{s} = \SI{91}{GeV}$ betrug, 
erhalten wir eine berechnete Kopplungskonstante
\begin{equation}
  \begin{split}
  \alpha_{s, \rm theo} &= \frac{12 \pi}{33 - 2 \cdot 5 \cdot \log\left(\frac{\SI{91}{GeV}^2}{\SI{0,2}{GeV}^2}\right)}\\
  &= \SI{0,133}{}~.
    \end{split}
\end{equation}

Die Abweichung von unserer gemessenen Kopplungskonstante liegt dabei unter einer Standardabweichung,
was unser Ergebnis glaubwürdig erscheinen lässt.

\section{Bestimmung der Anzahl der Neutrinogenerationen}
\label{sec:neutrinogenerationen}
Die Anzahl der Neutrinogenerationen $N_\nu$ lässt sich berechnen über
Gleichung~10 aus der Versuchsvorbereitung:

\begin{equation}
  N_\nu = \frac{\Gamma_{\rm inv}}{\Gamma_{\nu_e\nu_e}^{SM}}~.
\end{equation}

Dabei ist $\Gamma_{\nu_e\nu_e}^{SM} = \SI{166,1}{\mega\electronvolt}$ die im Standardmodell
vorhergesagte Zerfalls für Neutrinos unter Annahme einer Neutrinofamilie,
 ${\Gamma_{\rm inv}}$ dagegen ist die gemessene unsichtbare Breite der Neutrinozerfälle.

Sie lässt sich mit Gleichung~11 aus der Versuchsvorbereitung bestimmen über
\begin{equation}
\label{eq:gamma_inv}
\Gamma_{\rm inv} = \Gamma_{\rm tot} - \Gamma_{\rm had} - 3 \Gamma_{l\bar{l}}~,
\end{equation}

wobei $\Gamma_{\rm tot}$ die aus der Literatur entnommene gesamte Zerfallsbreite des $\PZzero$ ist.

Der Wirkungsquerschnitt für den Zerfall in ein Quarkpaar auf der Z-Resonanz, ist
\begin{equation}
\sigma_{\mathrm{had}} = \frac{12 \uppi \cdot \Gamma_{l \bar{l}} \cdot \Gamma_{\mathrm{had}}}{m_{\mathrm{Z}}^2 \cdot \Gamma_{\mathrm{tot}} } ~.
\end{equation}
Dies lässt sich auf die gesamte Zerfallsbreite Umstellen zu
\begin{equation}
\Gamma_{\mathrm{tot}} = \sqrt{\frac{12 \uppi \cdot \Gamma_{l \bar{l}} \cdot \Gamma_{\mathrm{had}}}{m_{\mathrm{Z}}^2 \cdot \sigma_{\mathrm{had}} }} ~.
\end{equation}

Der Wert für $\Gamma_{\mathrm{had}}=\SI{2.3+-0.3}{\MeV}$ wurde Aufgabe 1 entnommen.
$\Gamma_{l \bar{l}} = \SI{83.83}{\MeV}$ und $m_{\mathrm{Z}} = \SI{91.187}{\GeV}$ wurden im Blauen Buch [\ref{ref:BB}] angegeben.
Der Wirkungsquerschnitt ist $\sigma_{\mathrm{had}} = \frac{N_{\mathrm{had}}}{L}$, mit der Luminosität $L = \SI{28.48}{nb^{-1}}$.
Um barn in MeV umzurechnen wurde $\SI{1}{b}= \SI{0.00257}{ \MeV^{-2}}$ verwendet.
Die Zahl der hadronischen Zerfälle $N_{\mathrm{had}} = 450$ über alle 499 betrachteten Ereignisse wurde auf 1000 extrapoliert, 
da die Luminosität auch für 1000 Ereignisse gegeben war.
Mit einem Python-Skript und dem Paket Numpy wurde die Zerfallsbreite errechnet als

\begin{equation}
\Gamma_{\mathrm{tot}} = \SI{3.9+-0.5}{\GeV} ~.
\end{equation}

Der Fehler aus Aufgabe~1 pflanzte sich hier auch wieder fort.
Der Literaturwert liegt bei $\SI{2.5}{\GeV}$ [\ref{ref:pdg14}], ist also um 36~\% kleiner.
Gründe für die Abweichung sind in Aufgabe 1 erklärt.

Um die Anzahl der Neutrinofamilien zu bestimmen wurde mit Gleichung~\eqref{eq:gamma_inv} die Breite der unsichtbaren Neutrinozerfälle bestimmt mit
\begin{equation}
\Gamma_{\rm inv} = \SI{1.3+-0.2}{\GeV} ~.
\end{equation}

Das Verhältnis mit $\Gamma_{\nu_e\nu_e}^{SM} = \SI{166.1}{\MeV} $ ist ergab die Anzahl der Neutrinofamilien
\begin{equation}
 N_\nu = \SI{7.6+-1}{} ~.
\end{equation}

Dies weicht auch von den erwarteten 3 Familien ab.
Gründe für die Abweichung sind in Aufgabe~1 erläutert.


\section{Quellen}
\begin{enumerate}
\item Blaues Buch \label{ref:BB}
\item \url{http://home.web.cern.ch/about/accelerators/large-electron-positron-collider}
 (\today) \label{ref:cernlep}
% Quelle fuer PDG-Angaben: (noch nicht genutzt, daher auskommentiert)
\item K.A. Olive et al. (Particle Data Group), Chin. Phys. C, 38, 090001 (2014). \label{ref:pdg14}
\item \url{http://hands-on-cern.physto.se/hoc_v21en/index.html} (\today)\label{ref:hands-on}
\item \url{http://hyperphysics.phy-astr.gsu.edu/hbase/forces/couple.html} (\today) \label{ref:hyperphysics}
\end{enumerate}



\end{document}
