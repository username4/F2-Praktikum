\documentclass[a4paper,ngerman]{scrartcl}

\usepackage{amsmath}
\usepackage{amsfonts}
\usepackage{amssymb}
\usepackage[utf8]{inputenc}
\usepackage{graphicx}
\usepackage[ngerman]{babel}
\usepackage{hyperref}
\usepackage{float}
\usepackage{caption}
\usepackage{subcaption}
\usepackage{multirow}  %for tables
\usepackage{icomma} % Handle german comma as decimal point in numbers
\usepackage{units,siunitx} % Write units with correct spacing
\usepackage{upgreek} % provide non-italic greek letters
\usepackage{url}
\usepackage{hepnames} % hier gibt es symbole fuer unterschiedliche elementarteilchen, aber dieses paket gibts nicht im poolraum
\usepackage{booktabs}
%\usepackage{subfig}

% Formatting of table & figure captions
\captionsetup{font={sf,footnotesize},labelfont=bf,textfont=sl,skip=6pt}

\sisetup{locale = DE, % use "," as decimal point instead of "."
  exponent-product={\cdot},% used \cdot in front of 10^x
  separate-uncertainty} % give out uncertainty with \pm instead of in brackets

\setlength{\abovecaptionskip}{6pt}
\setlength{\belowcaptionskip}{0pt}

\title{Elementarteilchen 2\\Auswertung}
\date{\today}
\author{Michel Rausch, Michael Eliachevitch}

\begin{document}

\maketitle
\tableofcontents
\newpage

% Aufgaben
% - Bestimmung der Verzweigungsverhältnisse beim Z°-Zerfall und der Anzahl der
% verschiedenen Farbladungen der starken Wechselwirkung.
% - Bestimmung der Kopplungskonstanten der starken Wechselwirkung α s mit Hilfe der 3-
% Jet-Rate bei Erzeugung und Zerfall des Z-Bosons.
% - Bestimmung der Anzahl der Neutrino-Generationen im Standardmodell mittels der
% Zerfallsbreite des Z°.

\section{Verzweigungsverhältnisse beim \PZzero-Zerfall und Anzahl der Farbladungen}
\label{sec:verzweigungen}

\begin{table}
\centering
\caption{\textbf{Anzahl der Ereignisse.}}
\begin{tabular}{cccccc}
\toprule
Messreihe			&	\Pelectron\APelectron	&	\Ptauon\APtauon	&	\Pmuon\APmuon	&	3-Jet	&	2-Jet	\\
\midrule
1-100				& 	8	&	3	&	5	&	33	&	51	\\
101-200 (ohne 197)	& 	2	&	2	&	2	&	35	&	58	\\
201-300				&	3	&	1	&	2	&	36	&	58	\\
301-400				&	7	&	1	&	3	&	34	&	55	\\
401-500				&	5	&	2	&	3	&	37	&	53	\\
Gesamt				&	25	&	9	&	15	&	175	&	275	\\
Anteil				&5,0\%	&1,8\%  &3,0\%	&35,1\%	&55,1\%	\\
%(mapcar #'(lambda (x) (/ x 499.0)) '(25 9 15 175 275))
% (0.050100200400801605 0.018036072144288578 0.03006012024048096 0.35070140280561124 0.5511022044088176)
\bottomrule
\end{tabular}
\label{tab:count}
\end{table}

Es wurden Ereignisse am LEP aus der Onlinequelle \ref{ref:hands-on} mithilfe des webbasiertem Programms analysiert.
Die Daten Lagen in Messreihen von 100 Ereignissen vor.
Aufgeteilt wurde nach \Pelectron\APelectron ,	\Ptauon\APtauon	,	\Pmuon\APmuon , hadronischen 2-Jet-Ereignissen, sowie hadronischen 3-und Mehr-Jet-Ereignissen.
Bis auf dem Ereignis 197 konnten alle Ereignisse identifiziert werden.
Für jede Messreihe ist die Anzahl der unterschiedlichen Ereignisse in Tabelle~\ref{tab:count} aufgelistet.

Das Das Verhältnis von hadronischen zu leptonischen Zerfällen $R$ wurde aus Gleichung~(6) bestimmt.
Mit der ersten Messreihe wurde abgeschätzt, wie viele Ereignisse untersucht werden mussten.
In der ersten Reihe wurden 16 leptonische und 84 hadronische Ereignisse gezählt.
Wird die Zahl extrapoliert mit einem Faktor $m$, ist der zu erwartende relative Fehler auf 
\begin{equation}
\sqrt{\frac{1}{n \cdot m}} ~.
\end{equation}
$n$ ist die Zahl der leptonischen Ereignisse, $m$ gibt die Anzahl der Messreihen an. 
Die Zahl der hadronischen Ereignisse ist $100 - n $, daher ist $R$ nur von $n$ abhängig. 
Damit der relative Fehler unter 15~\% ist, muss $m$ größer als $2,78$ sein.
Es mussten also mindestens 3 Messreihen aufgenommen werden.

\clearpage

\section{Bestimmung der Kopplungskonstanten der starken Wechselwirkung}
\label{sec:kopplungskonstante}

Die Kopplungskonstante $\alpha_s$ der starken Wechselwirkung kann
mithilfe von Gleichung~(9) aus der Versuchsvorbereitung bestimmt
werden. Wir haben $N_3(y > y_{\rm cut}) = 175$ 3-Jets gemessen, von
$N_{\rm had} = 450$ Jet-Ereignissen insgesamt, und verwenden eine für
DELPHI bestimmte Konstante $C(y > y_{\rm cut} = 2{,}72$~\ref{ref:BB},
womit die Gleichung~(9) folgendes ergibt:

% (/ 175 (* 450 2.72)) 0.14297385620915032
\begin{equation}
  \frac{N_3(y > y_{\rm cut})}{C(y > y_{\rm cut})\cdot N_{\rm had}} = \frac{175}{2{,}72 \cdot 450} = 0,14 \approx \alpha_s~.
\end{equation}

Der Fehler hierfür folgt über aus dem ausschließlich relativen Fehler von $N_3$, da $N_2$ über $N_2 = N_h - N_3$ 
vollständig mit $N_3$ korreliert ist, wenn man $N_h$ als fest annimmt. 
Damit erhalten wir einen Fehler 

(/ 0.143 (+ (sqrt 175) 0))
\begin{equation}
  \Delta \alpha_s = \frac{\alpha_s}{\sqrt{N_3}} = 0,01~.
\end{equation}

Somit erhalten wir die Kopplungskonstante der starken Kernkraft

\begin{equation}
  \alpha_s = \SI{0,14 +- 0,01}{}~.
\end{equation}

Die starke Kopplungskonstante ist energieabhängig.
Sie sinkt mit dem Logarithmus der Schwerpunktsenergie $\sqrt{s}$ und lässt sich über
folgende empirische Formel aus der Literatur~\ref{ref:hyperphysics} berechnen:
\begin{equation}
  \alpha_s = \frac{12 \pi}{33 - 2 \cdot (N_u + N_f) \cdot \log\left(\frac{s}{\Lambda^2}\right)}~,
\end{equation}
wobei der empirische Parameter $\Lambda = \SI{0.2}{\giga\electronvolt}$ ist. 
Da die Schwerpunktsenergie unserer Ereignisse $\sqrt{s} = \SI{91}{GeV}$ betrug, 
erhalten wir eine berechnete Kopplungskonstante
\begin{equation}
  \begin{split}
  \alpha_{s, \rm theo} &= \frac{12 \pi}{33 - 2 \cdot 5 \cdot \log\left(\frac{\SI{91}{GeV}^2}{\SI{0,2}{GeV}^2}\right)}\\
  &= \SI{0,133}{}~.
    \end{split}
\end{equation}

Die Abweichung von unserer gemessenen Kopplungskonstante liegt dabei unter einer Standardabweichung,
was unser Ergebnis glaubwürdig erscheinen lässt.
\section{Bestimmung der Anzahl der Neutrinogenerationen}
\label{sec:neutrinogenerationen}





\section{Quellen}
\begin{enumerate}
\item Blaues Buch \label{ref:BB}
\item \url{http://home.web.cern.ch/about/accelerators/large-electron-positron-collider}
 (\today) \label{ref:cernlep}
% Quelle fuer PDG-Angaben: (noch nicht genutzt, daher auskommentiert)
% \item K.A. Olive et al. (Particle Data Group), Chin. Phys. C, 38, 090001 (2014). \label{ref:pdg14}
\item \url{http://hands-on-cern.physto.se/hoc_v21en/index.html} (\today)\label{ref:hands-on}
\item \url{http://hyperphysics.phy-astr.gsu.edu/hbase/forces/couple.html} (\today) \label{ref:hyperphysics}
\end{enumerate}



\end{document}
