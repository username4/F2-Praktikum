\documentclass[a4paper,ngerman]{scrartcl}

\usepackage{amsmath}
\usepackage{amsfonts}
\usepackage{amssymb}
\usepackage[utf8]{inputenc}
\usepackage{graphicx}
\usepackage[ngerman]{babel}
\usepackage{hyperref}
\usepackage{float}
\usepackage{caption}
\usepackage{subcaption}
\usepackage{multirow}  %for tables
\usepackage{icomma} % Handle german comma as decimal point in numbers
\usepackage{units,siunitx} % Write units with correct spacing
\usepackage{upgreek} % provide non-italic greek letters
\usepackage{url}
\usepackage{hepnames} % hier gibt es symbole fuer unterschiedliche elementarteilchen, aber dieses paket gibts nicht im poolraum
\usepackage{booktabs}
%\usepackage{subfig}

% Formatting of table & figure captions
\captionsetup{font={sf,footnotesize},labelfont=bf,textfont=sl,skip=6pt}

\sisetup{locale = DE, % use "," as decimal point instead of "."
  exponent-product={\cdot},% used \cdot in front of 10^x
  separate-uncertainty} % give out uncertainty with \pm instead of in brackets

\setlength{\abovecaptionskip}{6pt}
\setlength{\belowcaptionskip}{0pt}

\title{Elementarteilchen 2\\Auswertung}
\date{\today}
\author{Michel Rausch, Michael Eliachevitch}

\begin{document}

\maketitle
\tableofcontents
\newpage

\begin{table}
\centering
\caption{\textbf{Anzahl der Ereignisse.}}
\begin{tabular}{cccccc}
\toprule
Messreihe			&	ee	&	tt	&	uu	&	3	&	2	\\
\midrule
1-100				& 	8	&	3	&	5	&	33	&	51	\\
101-200 (ohne 197)	& 	2	&	2	&	2	&	35	&	58	\\
201-300				&	3	&	1	&	2	&	36	&	58	\\
301-400				&	7	&	1	&	3	&	34	&	55	\\
401-500				&	5	&	2	&	3	&	37	&	53	\\
gesamt				&	25	&	9	&	15	&	175	&	275	\\
\bottomrule
\end{tabular}
\label{tab:count}
\end{table}



\section{Quellen}
\begin{enumerate}
\item Blaues Buch \label{ref:BB}
\item \url{http://home.web.cern.ch/about/accelerators/large-electron-positron-collider}
 (\today) \label{ref:cernlep}
% Quelle fuer PDG-Angaben: (noch nicht genutzt, daher auskommentiert)
% \item K.A. Olive et al. (Particle Data Group), Chin. Phys. C, 38, 090001 (2014). \label{ref:pdg14}
\item \url{http://hands-on-cern.physto.se/hoc_v21en/index.html} (\today)\label{ref:hands-on}
\end{enumerate}



\end{document}
