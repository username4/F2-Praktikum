\documentclass[a4paper,ngerman]{scrartcl}

\usepackage{amsmath}
\usepackage{amsfonts}
\usepackage{amssymb}
\usepackage[utf8]{inputenc}
\usepackage{graphicx}
\usepackage[ngerman]{babel}
\usepackage{hyperref}
\usepackage{float}
\usepackage{caption}
\usepackage{subcaption}
\usepackage{multirow}  %for tables
\usepackage{icomma} % Handle german comma as decimal point in numbers
\usepackage{units,siunitx} % Write units with correct spacing
\usepackage{upgreek} % provide non-italic greek letters
\usepackage{url}
\usepackage{booktabs} % zu benutzen fuer \toprule und \bottomrule bei tabellen
%\usepackage{subfig}

% Formatting of table & figure captions
\captionsetup{font={sf,footnotesize},labelfont=bf,skip=6pt}
\captionsetup[sub]{font={sf,footnotesize}} % setting for subcaptions
\sisetup{ locale = DE, % use "," as decimal point instead of "."
per-mode=fraction, % use fractions instead of ^{-1} when doing \si{... \per ...} 
exponent-product={\cdot},% used \cdot in front of 10^x
separate-uncertainty % give out uncertainty with \pm instead of in brackets
} 
\setlength{\abovecaptionskip}{6pt}
\setlength{\belowcaptionskip}{0pt}

\title{Black Lipid Membrane\\ Verbesserung}
\date{\today}
\author{Michel Rausch, Michael Eliachevitch}

\begin{document}

\maketitle
\tableofcontents
\newpage

\section{Vorbereitung}

\subsection{Ionentransport}

Es existieren verschiedene Modelle, die beschreiben, wie Gramicidin A in die Zellmembran integriert wird, welche in Abbildung \ref{fig:wirkmechanismus} aufgelistet sind. Auf die Einzelheiten wird nicht im Detail eingegangen.


%Punkt B neu schreiben, alle Punkte etwas ausformulieren.

\begin{figure}
\includegraphics[width=0.5\textwidth]{abbildungen/wirkmechanismus.png}
\caption{\textbf{Verschiedene Modelle zur Erklärung der Zerstörung einer Bakterienzelle mittels Peptid-Antibiotika [\ref{ref:mappe}].}\\
\textbf{A: Barrel Stave model,} 
Hydrophobe, helixförmige Monomere bilden Poren in der Zellmembran.
\textbf{B: Toroidial Wormhole model,}  
Porenformation nahe phosphatidylethanolamin, oder phosphatidylserin Membranen. 
\textbf{C: Carpet model}  
Teppich-(engl. carpet-)ähnliche Anlagerung auf Membran.
\textbf{D:  Detergent similar model,}
Bi- und Mizellen bilden Flächen auf der Membran, durch deren Amphiphilie wird die Membran durchlässig.
\textbf{E:  In-plane diffusion model,}
Verdünnung der Lipidschicht.
}
\label{fig:wirkmechanismus}
\end{figure}


\subsection{Einzelkanalentstehung}

%Folgender Teil besser formuliert
Nur Dimere des Gramicidin A führen zu einer Bildung von Ionenkanälen. 
Ist die Rate der Bildung der Dimere gleich der Rate ihrer Vernichtung, spricht man von einem Gleichgewicht.
Ohne äußere Einwirkung bleibt die Zahl der Kanäle also etwa gleich.
Die Bildung und Zerstörung sind zufällige Ereignisse, daher kommt es zu einer variierenden Anzahl an Poren in der Membran und somit zu einer Fluktuation der Leitfähigkeit. 
Bei einer geringen Konzentration sind nur einzelne Kanäle vorhanden, daher ist die Quantisierung des Stromes $I_M$ deutlich erkennbar.

Die Dimerasation kann durch die Gleichung,

\begin{equation}
G_1 + G_1 	\xtofrom[k_d]{k_r} G_2 ,
\end{equation}

beschrieben werden. $k_d$ ist hier die Rate der Dissoziation und $k_r$ die der Entstehung der Dimere, $G_1$ entspricht den Monomeren, $G_2$ den Dimeren. Die Dissoziation folgt einem exponentiellem Zerfallsgesetz, die der Entstehungsreaktion ist linear zum Quadrat der Konzentration. Aus dem Verlauf des Stromes über die zeit lässt sich die Zerfallsrate bestimmen.



\section{Auswertung}



\clearpage
\section{Quellen}
\begin{enumerate}
\item Vorbereitungsmappe \label{ref:mappe}
\item http://www.chemie.de \label{ref:chemie.de}
\item http://www.wolframalpha.com \label{ref:wolfram}
\end{enumerate}




\end{document}
