\documentclass[a4paper,ngerman]{scrartcl}

\usepackage{amsmath}
\usepackage{amsfonts}
\usepackage{amssymb}
\usepackage[utf8]{inputenc}
\usepackage{graphicx}
\usepackage[ngerman]{babel}
\usepackage{hyperref}
\usepackage{float}
\usepackage{caption}
\usepackage{subcaption}
\usepackage{multirow}  %for tables
\usepackage{icomma} % Handle german comma as decimal point in numbers
\usepackage{units,siunitx} % Write units with correct spacing
\usepackage{upgreek} % provide non-italic greek letters
\usepackage{url}
%\usepackage{subfig}

% Formatting of table & figure captions
\captionsetup{font={sf,footnotesize},labelfont=bf,textfont=sl,skip=6pt}
\setlength{\abovecaptionskip}{6pt}
\setlength{\belowcaptionskip}{0pt}

\title{Magnetisierung\\Versuchsauswertung}
\date{\today}
\author{Michel Rausch, Michael Eliachevitch}

\begin{document}

\maketitle
\tableofcontents
\newpage

\section{Kalibrierung des SQUIDs}

\subsection*{Mit der Kalibrierungsspule}

\subsection*{Mit der Nickelprobe}



\section{Magnetisierungsmessungen}


\subsection{Terbiumprobe bei senkrechtem Einbau}

\subsubsection*{Messung mit im Nullfeld gekühlter Probe ohne Magnetisierung}

\subsubsection*{Messung mit im Nullfeld gekühlter Probe mit Magnetisierung bei tiefer Temperatur und einem magnetfeld mit \SI{150}{G}}

\subsubsection*{Messung mit im Feld gekühlter Probe bei \SI{150}{G}}




\subsection{Terbiumprobe bei parallelem Einbau}

\subsubsection*{Messung mit im Feld gekühlter Probe bei \SI{50}{G}}

\subsubsection*{Messung mit im Feld gekühlter Probe bei \SI{100}{G}}

\subsubsection*{Messung mit im Feld gekühlter Probe bei \SI{150}{G}}



\subsection{Gadoliniumprobe}

\subsubsection*{Messung mit im Feld gekühlter Probe bei \SI{1000}{G}}



\section{Quellen}
\begin{enumerate}
\item Vorbereitungsmappe.\label{ref:mappe}
\item \url{http://hydrogen.physik.uni-wuppertal.de/hyperphysics/hyperphysics/hbase/solids/squid.html} (18.1.2015).\label{ref:wuppertal}
\item \url{http://jsq.apps-1and1.net/category/information/what-is-a-squid/} (18.1.2015).
\label{ref:jsq}
\end{enumerate}



\end{document}
